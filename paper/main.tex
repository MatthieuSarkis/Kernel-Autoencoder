\documentclass[conference]{IEEEtran}
%\IEEEoverridecommandlockouts
% The preceding line is only needed to identify funding in the first footnote. If that is unneeded, please comment it out.

\usepackage{cite}
\usepackage{amsmath}
\usepackage{amssymb}
\usepackage{amsfonts}
\usepackage{algorithmic}
\usepackage{graphicx}
\usepackage{textcomp}
\usepackage{xcolor}
\usepackage{hyperref}
\hypersetup{
    colorlinks=true,
    linkcolor=blue,
    filecolor=magenta,      
    urlcolor=teal,
    }

\begin{document}

%\title{Conference Paper Title*\\
%{\footnotesize \textsuperscript{*}Note: Sub-titles are not captured in Xplore and
%should not be used}
%\thanks{Identify applicable funding agency here. If none, delete this.}
%}

\title{Kernel Autoencoders}

\author{
    \IEEEauthorblockN{1\textsuperscript{st} Matthieu Sarkis}
    \IEEEauthorblockA{\textit{Department of Physics and Materials Science (University of Luxembourg)}\\
    Luxembourg City, Luxembourg \\
    matthieu.sarkis@uni.lu}

    \and

    \IEEEauthorblockN{2\textsuperscript{nd} ?}
}

\maketitle

\begin{abstract}

    Kernel Autoencoders

\end{abstract}

\begin{IEEEkeywords}

    Kernel

\end{IEEEkeywords}

\section*{To Do List}

    \begin{itemize}

        \item Make the model variational, hence generative
        \item Improve the performance
        \begin{itemize}
            \item Improve the time complexity
            \item Is the model actually solvable analytically?
        \end{itemize}
        \item Study various datasets: 
        \begin{itemize}
            \item mnist
            \item fashion-mnist 
            \item qm7x
        \end{itemize}
        \item Any connection to deep gaussian processes?
        \begin{itemize}
            \item \href{http://inverseprobability.com/talks/notes/deep-gaussian-processes-a-motivation-and-introduction.html}{Deep Gaussian processes tutorial}
            \item \href{http://gaussianprocess.org/gpml/chapters/RW.pdf}{Book}
            \item \href{http://inverseprobability.com/talks/notes/gpss-session-1.html}{Intro to Gaussian Processes}
            \item \href{https://link.springer.com/content/pdf/10.1007/s10994-018-5723-3.pdf}{Autoencoders?}
        \end{itemize}
        \item Insist on molecules? Separate paper?
        \begin{itemize}
            \item Autoencoder for efficient molecular representation
            \item Build generative model
        \end{itemize}
        \item Use quantum circuit to evaluate the kernels, separate paper?
        \item Document the code much better
        \item Hyperparameters tuning with grid search or Metropolis-Hastings

    \end{itemize}

\section{Introduction}

\section*{Acknowledgment}

\cite{IEEEhowto:IEEEtranpage}.

\bibliographystyle{IEEEtran}  
\bibliography{IEEEabrv, bibliography}

\vspace{12pt}

\end{document}





%%%%%%%%%%%%%

%\begin{table}[htbp]
%\caption{Table Type Styles}
%\begin{center}
%\begin{tabular}{|c|c|c|c|}
%\hline
%\textbf{Table}&\multicolumn{3}{|c|}{\textbf{Table Column Head}} \\
%\cline{2-4} 
%\textbf{Head} & \textbf{\textit{Table column subhead}}& \textbf{\textit{Subhead}}& \textbf{\textit{Subhead}} \\
%\hline
%copy& More table copy$^{\mathrm{a}}$& &  \\
%\hline
%\multicolumn{4}{l}{$^{\mathrm{a}}$Sample of a Table footnote.}
%\end{tabular}
%\label{tab1}
%\end{center}
%\end{table}

%\begin{figure}[htbp]
%\centerline{\includegraphics{fig1.png}}
%\caption{Example of a figure caption.}
%\label{fig}
%\end{figure}